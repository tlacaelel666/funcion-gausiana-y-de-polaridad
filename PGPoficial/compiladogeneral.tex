\documentclass{book} % 'report' o 'book' funcionan bien para esta estructura con capítulos

% PAQUETES ESENCIALES
\usepackage[utf8]{inputenc} % Codificación de caracteres (para tildes y ñ)
\usepackage[spanish]{babel}  % Soporte para el idioma español
\usepackage[a4paper, margin=2.5cm]{geometry} % Márgenes razonables

% PAQUETES MATEMÁTICOS Y FÍSICOS
\usepackage{amsmath} 
\usepackage{amsfonts} 
\usepackage{amssymb} 
\usepackage{bm}          % Para símbolos matemáticos en negrita
\usepackage{physics}     % Paquete muy útil para notación física (ket, bra, derivadas, etc.)

% PAQUETES DE FORMATO Y OTROS
\usepackage{adjustbox}
\usepackage{algorithm}
\usepackage{hyperref}    % Para enlaces y referencias internas
\hypersetup{
    colorlinks=true,
    linkcolor=blue,
    filecolor=magenta,      
    urlcolor=blue,
    pdftitle={Postulado de la Polaridad Gravitacional Probabilística},
    pdfauthor={Jacobo Tlacaelel Mina Rodríguez},
}
\usepackage{verbatim} % Para bloques de pseudocódigo

% COMANDOS PERSONALIZADOS
\newcommand{\bvec}[1]{\vec{\mathbf{#1}}} % Comando original para vectores, se puede usar \bm{} también

\begin{document}

\frontmatter
\title{Postulado de la Polaridad Gravitacional Probabilística con Análisis Topológico-Evolutivo}
\author{Jacobo Tlacaelel Mina Rodríguez}
\date{\today}
\maketitle

\tableofcontents

\mainmatter

%==================================================================
% CAPÍTULO 1: RESUMEN E INTRODUCCIÓN
%==================================================================
\chapter{Introducción}

\section*{Resumen}
Este trabajo postula la teoría de la Polaridad Gravitacional Probabilística (PGP), que propone una reinterpretación de la naturaleza fundamental del universo. En lugar de partículas estáticas, se plantea que la realidad emerge de una repulsión-vibración persistente que dota al vacío cuántico de una cualidad de des-polarización. La masa no es una propiedad intrínseca, sino un fenómeno emergente de la coherencia vibracional de excitaciones que interactúan con el campo de Higgs, dando lugar a la dualidad onda-partícula. Se establece un nexo directo entre la asimetría observada en la superposición de un qubit y la influencia gravitacional sobre su Hamiltoniano, reinterpretando la gravedad como una manifestación de la polarización de esta estructura vibracional. El formalismo se encapsula en un Lagrangiano integrado y una Ecuación Maestra Topológico-Evolutiva que describe la dinámica del sistema, vinculando la topología del espacio con la evolución de la densidad de probabilidad. Finalmente, se propone un método numérico de Flujo Laminar de Información, inspirado en la mecánica de fluidos, para resolver las ecuaciones de la teoría de manera estable y eficiente.

\section{Motivación y Contexto}
En la frontera de la física, nos encontramos hoy ante un enigma profundo: ¿cómo unificamos la mecánica cuántica que rige lo infinitamente pequeño con la gravedad que moldea el cosmos? Pero antes de eso, ¿qué es la masa? Las teorías actuales nos han llevado lejos, pero persisten las fisuras, los "ruidos" y las anomalías que señalan hacia una comprensión incompleta de la realidad. Se propone una reinterpretación radical, aunque dentro de un mismo marco: el universo no se construye sobre partículas fundamentales estáticas, sino sobre algo que identificamos como repulsión-vibración persistente a una velocidad inimaginable, que interpretamos como cuerdas, excitaciones o cuasi-partículas.

Esta no es una fuerza en el sentido clásico, sino la dinámica primordial que "llena" el vacío cuántico, dándole una cualidad que llamamos des-polarización. Este vacío no es inerte; es un medio vibrante, adaptable y extraordinariamente rico que, al influenciarse y masificar su frecuencia, entra en una cohesión despolarizada. Al sumar excitaciones y actuar de manera coherente, logran una interacción con el campo de Higgs, compactando en una masa efectiva en un punto particular.

\section{Evidencia Observacional y Preguntas Fundamentales}
Pensemos en la célebre Supernova Refsdal. Descubierta por el Telescopio Espacial Hubble en 2014, su luz nos llegó en múltiples imágenes debido al efecto de lente gravitacional del cúmulo de galaxias MACS J1149.5+2223. Mientras la física convencional lo explica por los caminos que la luz toma a través del espacio-tiempo curvado, en nuestro marco, la Supernova Refsdal se convierte en la evidencia de cómo el \textit{tejido probabilístico polarizado del vacío} reconfigura las trayectorias de la luz con diferentes \textit{tiempos de llegada}, revelando no solo una lente gravitacional convencional, sino una lente de probabilidad cósmica.

Lo que proponemos es un cambio de paradigma: la realidad fundamental es una danza de polarizaciones y despolarizaciones. Si el universo es un vasto océano de repulsión-vibración, y la masa y la gravedad son meras manifestaciones de su polarización, ¿qué otras propiedades fundamentales aún no reconocidas emergen de esta dinámica? La respuesta a esta pregunta no solo redibujaría nuestro mapa del cosmos, sino que nos invitaría a repensar la esencia misma de la existencia.

%==================================================================
% CAPÍTULO 2: FUNDAMENTOS DEL POSTULADO
%==================================================================
\chapter{Fundamentos de la Polaridad Gravitacional Probabilística (PGP)}
La PGP se fundamenta en la repulsión-vibración persistente como la dinámica primordial que da forma a la realidad.
\begin{itemize}
    \item $M_{\text{repul}}(\mathbf{r})$ es la función de repulsión espacial.
    \item $E_0 = mc^2$ conecta directamente con la equivalencia masa-energía.
\end{itemize}

\section{Vacío Cuántico como Medio Vibrante}
El vacío no es inerte, sino un \textbf{medio vibrante, adaptable y extraordinariamente rico} con cualidades de despolarización.

\subsection{Susceptibilidad Vibratoria del Vacío}
La capacidad del vacío para responder a las excitaciones se modela con una susceptibilidad vibratoria:
\begin{equation}
\chi_{\text{vib}}(\mathbf{r},t) = \chi_0 \left(1 + \sum_n A_n \cos(\omega_n t + \phi_n)\right)
\end{equation}

\subsection{Campo de Respuesta Vibratoria}
La respuesta del vacío a estas vibraciones se puede describir mediante un campo de respuesta:
\begin{equation}
G_{\text{vib}}(\mathbf{r},t) = G_0 \cdot \mathcal{R}_{\text{vib}}(\mathbf{r},t) \cdot \exp\left(-\frac{\abs{\mathbf{r}}^2}{2\sigma_{\text{vib}}^2}\right)
\end{equation}

\section{Coherencia Despolarizada y Masificación}
Cuando las excitaciones \textbf{influencian y masifican su frecuencia}, entran en \textbf{cohesión despolarizada}, alcanzando la misma frecuencia de onda.

\subsection{Condición de Coherencia Despolarizada}
La coherencia se alcanza cuando las frecuencias de las excitaciones se sincronizan:
\begin{equation}
\omega_{\text{exc},i} = \omega_{\text{exc},j} = \omega_{\text{coherente}} \quad \forall i,j
\end{equation}

\subsection{Función de Masificación por Coherencia}
La masa efectiva emerge como resultado de esta coherencia energética, materializando la relación de Einstein:
\begin{equation}
m_{\text{eff}}(\mathbf{r}) = \frac{\hbar\omega_{\text{coherente}}}{c^2} \cdot \abs{\Psi_{\text{coherente}}(\mathbf{r})}^2
\end{equation}
donde $E = \hbar\omega_{\text{coherente}}$ es la energía de las vibraciones coherentes.

\subsection{Interacción con el Campo de Higgs}
La masificación coherente interactúa con el Campo de Higgs para compactar la masa en un punto, creando la dualidad onda-partícula que observamos.

%==================================================================
% CAPÍTULO 3: MARCO TEÓRICO Y CONCEPTOS CLAVE
%==================================================================
\chapter{Marco Teórico y Conceptos de Base}
\label{sec:fundamentos}
Para desarrollar el formalismo de la PGP, nos apoyamos en una serie de conceptos y herramientas bien establecidos en la física y las matemáticas.

\section{La Función de Onda Gaussiana}
La función Gaussiana es fundamental para describir distribuciones localizadas, como paquetes de ondas o densidades de probabilidad.
\begin{equation}
    f(x) = A \exp\left(-\frac{(x - \mu)^2}{2\sigma^2}\right)
    \label{eq:gaussiana1D}
\end{equation}

\section{Superposición de un Estado en una Base Ortonormal}
Un estado cuántico $\ket{\psi}$ se expresa como una superposición lineal de los vectores de una base $\{\ket{u_j}\}$:
\begin{equation}
    \ket{\psi} = \sum_{j} c_j \ket{u_j}
    \label{eq:superposicion_general}
\end{equation}
Donde $c_j = \braket{u_j | \psi}$ son los coeficientes de amplitud complejos y la probabilidad es $P(u_j) = \abs{c_j}^2$.

\section{Permitividad y Permeabilidad: Propiedades del Espacio-Medio}
La permitividad eléctrica ($\varepsilon$) y la permeabilidad magnética ($\mu$) describen cómo un medio responde a los campos. La velocidad de la luz en el vacío, $c = 1/\sqrt{\mu_0 \varepsilon_0}$, depende de estas constantes fundamentales. En nuestra teoría, estas no son constantes, sino campos dinámicos.

\section{Modelo de Kuramoto: Sincronización y Coherencia}
El Modelo de Kuramoto describe cómo un sistema de osciladores acoplados puede sincronizarse espontáneamente, un análogo directo a nuestra noción de "cohesión despolarizada".
\begin{equation}
    \frac{d\theta_i}{dt} = \omega_i + \frac{K}{N} \sum_{j=1}^{N} \sin(\theta_j - \theta_i)
    \label{eq:kuramoto_main}
\end{equation}

\section{Ecuación de Continuidad y Potencial Cuántico}
La formulación hidrodinámica de la mecánica cuántica descompone la ecuación de Schrödinger en una ecuación de continuidad para la densidad de probabilidad $\rho = \abs{\Psi}^2$:
\begin{equation}
    \frac{\partial \rho}{\partial t} + \nabla \cdot \mathbf{J} = 0
    \label{eq:continuity_quantum}
\end{equation}
y una ecuación de movimiento influenciada por un \textbf{Potencial Cuántico} $Q$, que emerge de la forma de la función de onda:
\begin{equation}
    Q(\mathbf{r}, t) = -\frac{\hbar^2}{2m} \frac{\nabla^2 \abs{\Psi}}{\abs{\Psi}}
    \label{eq:quantum_potential}
\end{equation}

%==================================================================
% CAPÍTULO 4: FORMALISMO MATEMÁTICO DE LA PGP-TOPOLÓGICA
%==================================================================
\chapter{Formalismo Matemático de la PGP-Topológica}

\section{Lagrangiano Integrado de la PGP-Topológica}
La dinámica completa del sistema se deriva del Principio de Mínima Acción aplicado a una densidad Lagrangiana total $\mathcal{L}_{\text{total}}$.
\begin{equation}
    \mathcal{L}_{\text{total}}=\mathcal{L}_{\text{PGP}}+\mathcal{L}_{\text{topo}}+\mathcal{L}_{\text{bayes}}+\mathcal{L}_{\text{mahal}}
\end{equation}
\textbf{Componentes del Lagrangiano:}
\begin{description}
    \item[$\mathcal{L}_{\text{PGP}}$:] Describe la dinámica de las excitaciones, el vacío polarizable y los campos electromagnéticos, incluyendo sus acoplamientos.
    \item[$\mathcal{L}_{\text{topo}}$:] Acopla la dinámica con las propiedades geométricas del espacio (curvatura media $H$, característica de Euler $\chi_E$, etc.).
    \item[$\mathcal{L}_{\text{bayes}}$:] Incorpora una penalización basada en la verosimilitud de las desviaciones de los campos respecto a valores de referencia.
    \item[$\mathcal{L}_{\text{mahal}}$:] Penaliza las desviaciones de los estados del sistema respecto a un modelo, cuantificadas por la distancia de Mahalanobis.
\end{description}

\section{Ecuación Maestra Topológico-Evolutiva}
La evolución de la densidad de probabilidad $\rho(\mathbf{r},t)$ de las excitaciones se rige por una ecuación maestra que integra difusión, transporte, interacciones locales, forzamiento topológico y ruido.
\begin{equation}
    \frac{\partial\rho}{\partial t}=\nabla\cdot(\mathbf{D}\nabla\rho)-\nabla\cdot(\rho\bvec{v})+S(\rho,\nabla\rho)+\Gamma_{\text{topo}}(H,\chi_E,K_{\text{gauss}})+\eta_{\text{bayes}}
\end{equation}
\subsection{Análisis de los Componentes e Influencia}
A continuación, se detalla el papel de cada término en la ecuación maestra.

\begin{description}
    \item[Tensor de Difusión Adaptativo ($\mathbf{D}$):] (Influencia: 50/100) Dispersa la densidad de probabilidad. Su naturaleza adaptativa, `$\mathbf{D}=D_0+\alpha\exp(-\sigma^2d_M^2(\nabla\rho,\boldsymbol{\mu}_{\text{ref}}))$`, significa que la tasa de dispersión varía según qué tan lejos esté el gradiente de densidad local de un estado de referencia, medido por la distancia de Mahalanobis $d_M^2$. Esto permite que la difusión sea no uniforme y anisotrópica.

    \item[Campo de Velocidad Topológica ($\bvec{v}$):] (Influencia: 50/100) Transporta la densidad de probabilidad de forma coherente. El campo de velocidad, `$\bvec{v}=-\nabla(\phi+\psi_{\text{topo}})$`, es impulsado por gradientes de potenciales, incluyendo un potencial topológico $\psi_{\text{topo}}$ que guía el flujo a lo largo de las características geométricas del espacio.

    \item[Término Fuente/Sumidero ($S(\rho,\nabla\rho)$):] (Influencia: 50/100) Modela la creación o aniquilación de excitaciones (y por tanto de probabilidad). Permite que la masa de probabilidad total del sistema no se conserve, reflejando procesos de intermasificación o despolarización.

    \item[Forzamiento Topológico ($\Gamma_{\text{topo}}$):] (Influencia: 30/100) Introduce una influencia directa de la geometría del espacio (Curvatura Media $H$, Característica de Euler $\chi_E$, etc.) en la evolución de $\rho$. Este término esculpe la densidad de probabilidad, pudiendo concentrarla o repelerla de regiones con características topológicas específicas.

    \item[Ruido Bayesiano ($\eta_{\text{bayes}}$):] (Influencia: 20/100) Modela las fluctuaciones inherentes del vacío y la incertidumbre. Sigue una distribución normal `$\eta_{\text{bayes}}\sim\mathcal{N}(0,\Sigma_M^{-1})$`, donde la matriz de covarianza $\Sigma_M$ define la magnitud y correlación de las fluctuaciones aleatorias.
\end{description}

%==================================================================
% CAPÍTULO 5: APLICACIONES Y MODELADO
%==================================================================
\chapter{Aplicaciones y Modelado de la Polaridad}

\section{Adaptación de las Ecuaciones de Maxwell}
En el marco PGP, la permitividad $\varepsilon$ y la permeabilidad $\mu$ no son constantes, sino \textbf{tensores} $\boldsymbol{\varepsilon}(\mathbf{r}, t)$ y $\boldsymbol{\mu}(\mathbf{r}, t)$ que dependen de la polarización del vacío. Las relaciones constitutivas se generalizan a $\bvec{D} = \boldsymbol{\varepsilon} \cdot \bvec{E}$ y $\bvec{H} = \boldsymbol{\mu}^{-1} \cdot \bvec{B}$.

\section{Ecuación de Ginzburg-Landau para la Transición de Polarización}
La emergencia de un estado polarizado desde un estado desordenado puede modelarse con un parámetro de orden complejo $\Psi(\mathbf{r}, t)$ que sigue una ecuación de Ginzburg-Landau:
\begin{equation}
    \pdv{\Psi}{t} = \alpha \Psi - \beta \abs{\Psi}^2 \Psi + D \nabla^2 \Psi
    \label{eq:ginzburg_landau}
\end{equation}
Un cambio de signo en $\alpha$ puede desencadenar la transición de fase hacia la polarización ($\Psi \neq 0$).

\section{Parámetro de Masa-Gravedad y Potencial Efectivo}
Definimos un \textbf{Parámetro de Masa-Gravedad} $\bvec{P}_{\text{MG}}$ que emerge de los gradientes en la polarización del vacío (representada por un índice de refracción efectivo $n(\mathbf{r})$):
\begin{equation}
    \bvec{P}_{\text{MG}}(\mathbf{r}) = \chi(\mathbf{r}) \frac{\nabla n(\mathbf{r})}{\abs{\nabla n(\mathbf{r})}}
    \label{eq:pmg_definition}
\end{equation}
Este parámetro genera un \textbf{Potencial Efectivo de Masa-Gravedad} $U_{\text{MG}}$ que afecta la dinámica de las partículas y campos:
\begin{equation}
    U_{\text{MG}}(\mathbf{r}) = \bvec{P}_{\text{MG}}(\mathbf{r}) \cdot \bvec{B}_{\text{eff}}(\mathbf{r})
    \label{eq:umg_definition}
\end{equation}

%==================================================================
% CAPÍTULO 6: METODOLOGÍA NUMÉRICA
%==================================================================
\chapter{Método Numérico de Flujo Laminar de Información}
Para resolver las complejas ecuaciones de la PGP, proponemos un enfoque numérico inspirado en la mecánica de fluidos computacional (CFD).

\section{Conceptualización del Flujo Laminar de Información}
En la PGP, mapeamos el flujo laminar de un fluido a capas de información cuántica que evolucionan suavemente. Esta analogía permite diseñar algoritmos numéricamente estables y físicamente intuitivos. Las "líneas de corriente" del fluido corresponden a trayectorias de máxima coherencia o probabilidad.

\section{Implementación del Método}
El método se basa en varios pilares:
\begin{itemize}
    \item \textbf{Discretización Espacial Adaptativa:} La malla computacional se refina en regiones de altos gradientes (transiciones de fase, frentes de onda), similar al AMR en CFD.
    \item \textbf{Integración Temporal IMEX (Implícito-Explícito):} Se tratan los términos de evolución unitaria (rápidos) de forma explícita y los términos disipativos o de decoherencia (rígidos) de forma implícita, garantizando la estabilidad.
    \item \textbf{Algoritmos Bayesianos con Dinámicas de Flujo:} Se utilizan filtros de partículas donde las partículas (hipótesis sobre el estado del sistema) son "advectadas" a lo largo de las líneas de corriente del flujo de información.
\end{itemize}

\subsection{Algoritmo Principal en Python (Conceptual)}
\begin{verbatim}
class PGPFluidSolver:
    def __init__(self, domain, initial_conditions):
        self.domain = domain
        self.psi_field = initial_conditions['wavefunction']
        # ... inicializar otros campos: D, v, S, etc.

    def evolve_system(self, t_final, dt):
        """Evolución completa del sistema PGP con flujo laminar."""
        t = 0
        while t < t_final:
            # 1. Actualizar campos de flujo (v, D) basados en psi_field.
            self.update_flow_fields()
            
            # 2. Advectar/Difundir el campo psi_field (Paso IMEX).
            self.evolve_master_equation(dt)

            # 3. Aplicar forzamiento topológico y términos fuente.
            self.apply_forcing_and_sources(dt)

            # 4. Refinar malla si es necesario.
            if self.needs_refinement():
                self.refine_mesh_following_flow()
            t += dt
        return self.extract_physical_quantities()
\end{verbatim}

\section{Ventajas del Enfoque}
\begin{enumerate}
    \item \textbf{Estabilidad Numérica Natural:} La analogía con el flujo ayuda a evitar inestabilidades al respetar una condición CFL (Courant-Friedrichs-Lewy) de forma inherente.
    \item \textbf{Eficiencia Computacional:} El refinamiento adaptativo concentra el esfuerzo computacional solo donde es necesario.
    \item \textbf{Preservación de Propiedades Físicas:} La causalidad y la localidad se pueden preservar por construcción en el algoritmo.
\end{enumerate}

%==================================================================
% CAPÍTULO 7: CONCLUSIONES
%==================================================================
\chapter{Conclusiones y Perspectivas Futuras}
Hemos presentado el postulado de la Polaridad Gravitacional Probabilística, una teoría que unifica la emergencia de la masa, la gravedad y la dinámica cuántica bajo un único principio: la polarización de un vacío vibracional. El formalismo matemático, centrado en un Lagrangiano topológico y una Ecuación Maestra evolutiva, proporciona un marco riguroso para describir la interacción entre la geometría del espacio y la materia.

El enfoque de flujo laminar de información ofrece una vía prometedora para la simulación numérica de este sistema, abriendo la puerta a la exploración de fenómenos como la formación de estructuras cósmicas, la decoherencia cuántica y las lentes gravitacionales desde una nueva perspectiva.

Las perspectivas futuras de este trabajo incluyen la derivación de soluciones analíticas en casos simplificados, la implementación completa del simulador de flujo laminar y la confrontación de sus predicciones con datos observacionales, tanto cosmológicos (como el fondo cósmico de microondas) como de laboratorio (en sistemas de computación cuántica). La PGP, de ser validada, podría representar un paso significativo hacia la comprensión de la naturaleza fundamental de la realidad.

\backmatter

%==================================================================
% BIBLIOGRAFÍA
%==================================================================
\begin{thebibliography}{99}
    \bibitem{source1} Universidad de Granada (ugr.es). *Material académico relevante*.
    \bibitem{source2} Universidad Autónoma Metropolitana (uam.mx). *Publicaciones y recursos*.
    \bibitem{source3} Wikipedia.org. *Artículos de referencia sobre física y matemáticas*.
    \bibitem{source4} Universidad Nacional Autónoma de México (unam.mx). *Investigación y docencia*.
    \bibitem{source5} Universidad de Castilla-La Mancha (uclm.es). *Recursos académicos*.
    \bibitem{source6} Flashcards.world. *Material de estudio general*.
    \bibitem{source7} Universidad de Valladolid (uva.es). *Documentación académica*.
    \bibitem{source8} Infomed (sld.cu). *Publicaciones científicas*.
    \bibitem{source9} Universidad de Buenos Aires (uba.ar). *Repositorio institucional*.
    \bibitem{source10} Sociedad Matemática Mexicana (smm.org.mx). *Publicaciones especializadas*.
    \bibitem{source11} Universidad Nacional de Colombia (unal.edu.co). *Material de investigación*. 
    \bibitem{source12} Física del bosón de
Higgs en el LHC
Patricia Castiella Esparza
\end{thebibliography}

\end{document}